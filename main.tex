%%%%%%%%%%%%%%%%%%%%%%%%%%%%%%%%%%%%%%%%%
% "ModernCV" CV and Cover Letter
% LaTeX Template
% Version 1.1 (9/12/12)
%
% This template has been downloaded from:
% http://www.LaTeXTemplates.com
%
% Original author:
% Xavier Danaux (xdanaux@gmail.com)
%
% License:
% CC BY-NC-SA 3.0 (http://creativecommons.org/licenses/by-nc-sa/3.0/)
%
% Important note:
% This template requires the moderncv.cls and .sty files to be in the same 
% directory as this .tex file. These files provide the resume style and themes 
% used for structuring the document.
%
%%%%%%%%%%%%%%%%%%%%%%%%%%%%%%%%%%%%%%%%%

%----------------------------------------------------------------------------------------
%	PACKAGES AND OTHER DOCUMENT CONFIGURATIONS
%----------------------------------------------------------------------------------------

\documentclass[11pt,a4paper,roman]{moderncv} % Font sizes: 10, 11, or 12; paper sizes: a4paper, letterpaper, a5paper, legalpaper, executivepaper or landscape; font families: sans or roman

\moderncvstyle{classic} % CV theme - options include: 'casual' (default), 'classic', 'oldstyle' and 'banking'
\moderncvcolor{black} % CV color - options include: 'blue' (default), 'orange', 'green', 'red', 'purple', 'grey' and 'black'

\usepackage{lipsum} % Used for inserting dummy 'Lorem ipsum' text into the template

\usepackage[scale=0.85]{geometry} % Reduce document margins
%\setlength{\hintscolumnwidth}{3cm} % Uncomment to change the width of the dates column
%\setlength{\makecvtitlenamewidth}{6.0cm} % For the 'classic' style, uncomment to adjust the width of the space allocated to your name

%----------------------------------------------------------------------------------------
%	NAME AND CONTACT INFORMATION SECTION
%----------------------------------------------------------------------------------------

\firstname{Gonzalo Ra\'ul}% Your first name
\familyname{Olave  Wolff} % Your last name

% All information in this block is optional, comment out any lines you don't need
\title{Curriculum Vitae}
\address{Las encinas 64, dpto 412}{Estaci\'on Central, RM, Chile}
\mobile{(+569) 97736166}
%\phone{(+562) 23147445}
\email{growolff@gmail.com}
%\homepage{growolff.github.io} % The first argument is the url for the clickable link, the second argument is the url displayed in the template - this allows special characters to be displayed such as the tilde in this example
%\extrainfo{Nacionalidad Chilena}
%\photo[70pt][0.4pt]{GonzaloOlave} % The first bracket is the picture height, the second is the thickness of the frame around the picture (0pt for no frame)

%----------------------------------------------------------------------------------------

\begin{document}

\makecvtitle % Print the CV title
\section{Resumen}
Ing. el\'elctrico y MSc. en Ing. El\'ectrica, con intereses en la investigaci\'on, desarrollo y aplicaci\'on de nuevas tecnolog\'ias aplicadas a rob\'otica, IoT y sistemas aut\'onomos inteligentes. Curioso de las posibilidades de la automatizaci\'on de nuestras vidas y como afectan nuestras relaciones interpersonales. Diverso en capacidades orientadas a la resoluci\'on de problemas aplicados a la ingenier\'ia el\'ectrica, mecatr\'onica, ciencias de la computaci\'on y otras \'areas afines. Tengo experiencia en desarrollo y c\'atedra de cursos de Rob\'otica con ROS, Arduino, IoT y fabricaci\'on digital.
%Actualmente realizo mi tesis de Mag\'ister en Ingenier\'ia Civil El\'ectrica en el dise\~no y fabricaci\'on de una mano rob\'otica que utiliza un sistema de actuaci\'on por cuerdas trenzadas, que al imitar catacter\'isticas de tendones y m\'usculos animales permite la manipulaci\'on de multiples objetos.
Creo que la sinceridad y honestidad son necesarias para formar buenas relaciones y sana convivencia en comunidad.


%----------------------------------------------------------------------------------------
%	EDUCATION SECTION
%----------------------------------------------------------------------------------------

\section{Estudios}

\cventry{2022}{Mag\'ister en Ciencias de la Ingenier\'ia menci\'on El\'ectrica}{}{Universidad de Chile}{}{Tesis titulada 'Dise\~no, fabricaci\'on y an\'alisis de una mano rob\'otica sub-actuada utilizando actuadores de cuerda trenzada'. Graduado con distinci\'on m\'axima.}
\cventry{2022}{Ingenier\'ia Civil El\'ectrica}{Rob\'otica e Inteligencia Computacional}{Universidad de Chile}{}{}

\cventry{2014}{Licenciado en Ciencias de la Ingenier\'ia menci\'on Ingenier\'ia El\'ectrica}{}{Universidad de Chile}{}{}  % Arguments not required can be left empty

%\cventry{}{Educaci\'on Secundaria}{Instituto Art\'istico secundario de la Universidad de Chile - ISUCH (7mo a 3ro medio), Liceo Augusto D'Halmar (4to medio)}{}{}{}

%----------------------------------------------------------------------------------------
%	WORK EXPERIENCE SECTION
%----------------------------------------------------------------------------------------

\section{Experiencia laboral}

%----------------------------------------------------------------------------------------
%\subsection{Experiencia laboral}

\cventry{2018--2023}{Ingeniero el\'ectrico y Coordinador de Comunidad en Fablab U. de Chile}{}{}{}{- Dise\~no y fabricaci\'on de circuitos electr\'onicos, robots, sistemas de control, m\'aquinas CNC, y otros aparatos electr\'onicos. (https://gitlab.com/fablab-u-de-chile) \\
- Asesor\'ia en desarrollo electr\'onico a emprendimientos de Hardware con base cient\'ifica tecnol\'ogica. Desarrollo de Ventilador Mec\'anico durante primer a\~no de pandemia. \\
- Desarrollador y profesor de los talleres: \textit{Introducci\'on a Arduino}, \textit{Fabricaci\'on de PCBs}, \textit{ Introducci\'on al IoT}, \textit{Documentaci\'on de Proyectos de Dise\~no Abierto} y del Programa Formativo BRC. \\
- Coordinador de comunidad FabLab, organizaci\'on de eventos de difusi\'on y organizaci\'on del Festival de rob\'otica universitaria Beauchef Robotics Challenge 2018, 2019 y 2022. }

%\cventry{2017}{Chief Operations Officer en Duckietown Engineering Chile}{}{}{}{COO de la empresa ficticia Duckietown Enginerring Chile, utilzada para implementar el curso EI2001 Taller de Proyecto: \textit{Duckietown: Desarrollando veh\'iculos aut\'onomos.}  }

\cventry{2015-2018}{Desarrollador en KnightRobotics}{}{}{}{Fabricaci\'on de robots educativos. Modelamiento 3D. Desarrollo de capacitaciones y clases de rob\'otica y Arduino para estudiantes de colegio y profesores}

\cventry{Dic 2013 a Feb 2014}{Pr\'actica profesional \#2}{AMTC - FCFM}{Universidad de Chile}{}{
Dise\~no e implementaci\'on de nuevo software para la interacci\'on mediante voz del robot \textit{Bender}. Expansi\'on de las capacidades de manipulaci\'on del robot agregando un brazo al nivel del suelo.}

\cventry{2013}{Ingeniero Freelance}{}{}{}{Construcci\'on, dise\~no y/o arreglo de maquinaria, sistemas aut\'onomos o proyectos electromec\'anicos varios. 
\begin{itemize}
\item Dise\~no y construcci\'on de un sistema de control de una moto-bomba para PYME dedicada al embotellamiento y transporte de agua.
%\item Sistema de regad\'io autom\'atico utilizando arduino.
\item Puesta en marcha de una Router CNC para Did\'acticos DaVinci, PYME de juegos did\'acticos educativos en madera.
\end{itemize}
}
\cventry{Ene 2013}{Pr\'actica profesional \#1}{AMTC - FCFM}{Universidad de Chile}{}{
Durante el periodo de la pr\'actica fueron dise\~nados y construidos ambos brazos del robot Bender. Se grab\'o y edit\'o el video de clasificaci\'on para la Robocup 2013}

%----------------------------------------------------------------------------------------
\section{Experiencia en Docencia}

\cventry{2020-2023}{CD2201 M\'odulo Interdisciplinario, Duckietown: Rob\'otica aplicada a veh\'iculos aut\'onomos}{FCFM}{Universidad de Chile}{}{Coordinador, desarrollador y profesor del curso.}
\cventry{2022}{ME6030 Manipuladores Rob\'oticos}{FCFM}{Universidad de Chile}{}{}
\cventry{Ago. 2019}{Workshop 'Applying EcoDesign approach in service value offer of FabLabs: seeking for a sustainable systemic innovation'}{Congreso Mundial de Fablabs - Fab15}{El Gouna, Egipto}{}{}
\cventry{2017-2019}{EI2001 Duckietown: Desarrollando veh\'iculos aut\'onomos}{FCFM}{Universidad de Chile}{}{}
\cventry{2016--2017}{Cursos de rob\'otica con Arduino en universidades y colegios}{}{}{}{}
%\cventry{2016--2017}{Introducci\'on a la Rob\'otica con Arduino}{SOFOFA}{}{}{Profesor del curso impartido por la empresa KnightRobotics en los liceos Benjam\'in D\'avila Larra\'in y Agust\'in Edwardss Ross.}
%\cventry{2016}{Taller de Rob\'otica}{Escuela de Verano}{Universidad de Chile}{}{Profesor del curso de rob\'otica de la escuela de verano que se imparte en la FCFM.}
%\cventry{2016}{Taller de Rob\'otica con Arduino}{PentaUC}{Pontificia Universidad Cat\'olica de Chile}{}{Profesor del taller de rob\'otica del Penta UC a ni\~nos de 7mo b\'asico.}
\cventry{2014--2017}{Mentor de Rob\'otica Educativa}{Funcaci\'on Mustakis}{}{}{Desarrollador y mentor del Programa de Rob\'otica Educativa impartido en la FCFM}

\cventry{2012-2013}{Ayudante de cursos FI1002, EL4002 y FI2003 de la FCFM}{}{}{}{}
%\cventry{2013}{FI2003 - M\'etodos Experimentales}{Auxiliar del profesor Rodrigo Espinoza G.}{}{}{}
%\cventry{2012}{EL4002 - Sistemas Digitales}{Prof. Ayudante del profesor Francisco Rivera}{}{}{}
%\cventry{2012}{FI1002 - Sistemas Newtonianos}{Prof. Ayudante del profesor Victor Fuenzalida E.}{}{}{}


%------------------------------------------------
\section{Congresos y cursos}

\cventry{2021}{General Course On Intellectual Property DL-101}{World Intellectual Property Organization - WIPO}{}{}{}

\cventry{Dic. 2015}{Voluntario en 'International Conference on Computer Vision (ICCV)' 2015}{}{}{}{Apoyo en el registro de asistentes y servicios en Workshops.}

\cventry{Oct. 2012}{Congreso Latinoamericano de Ingenier\'ia (CLI) Chile 2012}{Universidad T\'ecnica Feder\'ico Santa Mar\'ia}{Valpara\'iso}{}{}

\section{Experiencia Extracurricular}

\cventry{2016}{Comunidad de Rob\'otica UChile}{}{}{}{Fundador de la Comunidad de Rob\'otica de la Universidad de Chile. Organizaci\'on y participaci\'on en seminarios de rob\'otica, workshops de ROS, cursos de rob\'otica para escolares, competencia Beauchef Robotics Challenge (https://b-rc.cl) y el curso de 2do a\~no de Ingenier\'ia: Duckietown.}

\cventry{2011 a 2017}{Investigador Asistente}{Laboratorio de Rob\'otica, DIE, Universidad de Chile}{}{}{Miembro del equipo "UChile Homebreakers". Desarrollo de caracter\'isticas para el robot de servicio Bender y otros proyectos del laboratorio de Rob\'otica. Modelado 3D del robot, reconocimiento de voz, brazos antropom\'orficos, cuello y cabeza.}

\cventry{Jul. 2015}{Competidor de la liga @Home de la competencia internacional RoboCup}{}{}{}{Doceavo lugar en la liga @Home con el robot de servicio Bender, a\~no 2015. Anhui International Conference \& Exhibition Center, Hefei, Provincia de Anhui, China.}

\cventry{Jul. 2014}{Competidor de la liga @Home de la competencia internacional RoboCup}{}{}{}{S\'eptimo lugar en la liga @Home con el robot de servicio Bender, a\~no 2014. Centro de convenciones Poeta Ronaldo Cunha Lima, Jo\~ao Pessoa, Regi\'on de Para\'iba, Brasil.}

%\cventry{2010}{Voluntario del proceso de reconstrucci\'on para el terremoto del 27/F}{}{En las localidades de Empedrado y Parral}{Regi\'on del Maule, organizado por la FECH}{}

%\cventry{2008}{Voluntario en la Campa\~na Nacional de Alfabetizaci\'on "De Mart\'i a Fidel"}{}{En el municipio de San Rafael del Sur en Nigaragua}{}{}

%------------------------------------------------------------

\section{Publicaciones}

\cventry{2021}{Digital Biofabrication Node}{Danisa Peric, Joakin Ugalde, Joaquin Rosas, Victor Contreras, Gonzalo Olave, Mercedes Baldovino}{This is Distributed Design, pp. 194-201}{}{}

\cventry{2019}{Lector de CO2 para inclusiones fluidas}{Samanta Aravena-Gonz\'alez, Daniel Moncada, Victor Contreras, Gonzalo Olave}{1er Congreso de la Asociaci\'on Latinoamericana de Volcanolog\'ia}{}{Antofagasta, Chile}

\cventry{2017}{Aprendizaje Interdisciplinario en Rob\'otica: La Experiencia Innovadora de Duckietown Chile}{Mattamala, M., Olave, G., Campusano, M., G\'omez, C., Mart\'inez, L., Estef\'o, P., Ugalde, J., Urrutia, J., San-Mart\'in, F., Z\'u\~niga, P., Carrasco, J, Villar, C., Gonz\'alez, R.}{XXX Congreso SOCHEDI}{}{Santiago, Chile.}

\cventry{2017}{The NAO Backpack: An Open-hardware Add-on for Fast Software Development with the NAO Robot}{Mattamala, M., Olave, G., Gonz\'alez, C., Hasb\'un, N., Ruiz-del-Solar, J.}{RoboCup Symposium 2017}{}{Nagoya, Japan}

\cventry{2017}{UChile Homebreakers 2017 @Home League Team Description Paper}{Mart\'inez L., Mu\~noz R., Olave G., Pais G., D\'iaz G., G\'omez D., Campanini D., Orellana P., Loncomilla P., Ruiz-del-Solar J.}{In RoboCup 2017: Robot World Cup XXI}{}{Nagoya, Japan}

\cventry{2015}{UChile Homebreakers 2015 @Home League Team Description Paper}{Mart\'inez L., Pavez M., Olave G., Correa M., S\'anchez L., Loncomilla P., Ruiz-del-Solar J.}{In RoboCup 2015: Robot World Cup XIX}{}{Hefei, China}

\cventry{2014}{UChile Homebreakers 2014 @Home League Team Description Paper}{Correa M., Pavez M., Olave G., Tampier C., Retamal C., Pairo W., Bernuy F., Herrmann D., Verschae R., Loncomilla P. Mart\'inez L, Daud O., Ruiz-del-Solar J.}{In RoboCup 2014: Robot World Cup XVIII}{}{Joao Pessoa, Brasil}

%\cventry{2013}{UChile Homebreakers 2013 @Home League Team Description Paper}{Correa M., Pavez M., Olave G., Castro I., Tampier C., Retamal C., Pairo W., Verschae R., Loncomilla P., Ruiz-del-Solar J.}{In RoboCup 2013: Robot World Cup XVII}{}{Eindhoven, Holanda}


%----------------------------------------------------------------------------------------
%	COMPUTER SKILLS SECTION
%----------------------------------------------------------------------------------------
\section{Informaci\'on Adicional}

\subsection{\textbf{Manejo de software y lenguajes de programaci\'on}}
\cvitem{}{Git, ROS, Solidworks, Fusion 360, Autodesk Inventor, Eagle, Kicad, Matlab, Linux, Office, Illustrator, Premiere Pro, Arduino, Python, C, C++, \LaTeX}

%----------------------------------------------------------------------------------------
%	LANGUAGES SECTION
%----------------------------------------------------------------------------------------
\subsection{\textbf{Lenguajes}}
\cvitem{}{Espa\~nol nativo}
\cvitem{}{Ingl\'es avanzado, comprensi\'on, conversaci\'on, lectura y escritura}

%----------------------------------------------------------------------------------------
%	INTERESTS SECTION
%----------------------------------------------------------------------------------------

\subsection{\textbf{Intereses personales}}

\renewcommand{\listitemsymbol}{-~} % Changes the symbol used for lists

\cvlistdoubleitem{Rob\'otica \& IoT}{Cultura libre y conocimiento abierto}
\cvlistdoubleitem{Ciclismo}{Fotograf\'ia an\'aloga y digital}

\section{Referencias}

%\cventry{}{PhD. Javier Ruiz-Del-Solar}{}{\textsc{Associate Researcher, Robotics and Automation}}{Advanced Mining Tecnology Center, FCFM, Universidad de Chile}{mcorrea@amtc.cl}

\cventry{}{MSc. Danisa Peric}{Directora Ejecutiva del FabLab U. de Chile}{}{ FCFM, Universidad de Chile}{danisa@fablab.uchile.cl}
\cventry{}{PhD. Rub\'en Fern\'andez}{Profesor Asociado, Dpto. Ing. Mec\'anica. Materiales y M\'etodos de Manofactura}{}{DIMEC, FCFM, Universidad de Chile}{rufernan@uchile.cl}
\cventry{}{PhD. Mauricio Correa}{Investigador Asociado, Rob\'otica y Automatizaci\'on}{}{Advanced Mining Tecnology Center, FCFM, Universidad de Chile}{macorrea@amtc.cl}


\end{document}

